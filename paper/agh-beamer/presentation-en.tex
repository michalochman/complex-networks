% Copyright 2004 by Till Tantau <tantau@users.sourceforge.net>.
% Copyright 2009 by Barbara Filip <basiaf@fatcat.ftj.agh.edu.pl>
%
% In principle, this file can be redistributed and/or modified under
% the terms of the GNU Public License, version 2.
%
% However, this file is supposed to be a template to be modified
% for your own needs. For this reason, if you use this file as a
% template and not specifically distribute it as part of a another
% package/program, I grant the extra permission to freely copy and
% modify this file as you see fit and even to delete this copyright
% notice. 

\documentclass{beamer}


\mode<presentation>
{
  \usetheme{agh}
}

\usepackage[english]{babel}
% or whatever

\usepackage[latin1]{inputenc}
% or whatever

\usepackage{times}
\usepackage[T1]{fontenc}
% Or whatever. Note that the encoding and the font should match. If T1
% does not look nice, try deleting the line with the fontenc.

\title[Short Paper Title] % (optional, use only with long paper titles)
{Presentation Title}

\subtitle
{Presentation Subtitle} % (optional)

\author[Basia Filip] % (optional, use only with lots of authors)
{B.~Filip}
% - Give the names in the same order as the appear in the paper.
% - Use the \inst{?} command only if the authors have different
%   affiliation.

\institute[AGH] % (optional, but mostly needed)
{
  Faculty of Physics and Applied Computer Science\\
  AGH
}
% - Use the \inst command only if there are several affiliations.
% - Keep it simple, no one is interested in your street address.

\date[2009] % (optional, should be abbreviation of conference name)
{Summer project 2009}
% - Either use conference name or its abbreviation.
% - Not really informative to the audience, more for people (including
%   yourself) who are reading the slides online

\subject{Theoretical Computer Science}
% This is only inserted into the PDF information catalog. Can be left
% out. 



% If you have a file called "university-logo-filename.xxx", where xxx
% is a graphic format that can be processed by latex or pdflatex,
% resp., then you can add a logo as follows:

% \pgfdeclareimage[height=0.5cm]{university-logo}{university-logo-filename}
% \logo{\pgfuseimage{university-logo}}



% Delete this, if you do not want the table of contents to pop up at
% the beginning of each subsection:
% \AtBeginSubsection[]
% {
%  \begin{frame}<beamer>{Outline}
%    \tableofcontents[currentsection,currentsubsection]
%  \end{frame}
% }


% If you wish to uncover everything in a step-wise fashion, uncomment
% the following command: 

%\beamerdefaultoverlayspecification{<+->}


\begin{document}

\begin{frame}
  \titlepage
\end{frame}

\begin{frame}{Outline}
  \tableofcontents
  % You might wish to add the option [pausesections]
\end{frame}

\section{Punkt 1}

\subsection{Podstawy}

\begin{frame}{Tytul Tego Slajdu.}
  % - A title should summarize the slide in an understandable fashion
  %   for anyone how does not follow everything on the slide itself.

  \begin{itemize}
  \item
    Uzywaj \texttt{itemize} bo warto.
  \item
    Podpunkt krotkim zdaniem.
  \end{itemize}
\end{frame}

\begin{frame}{Kolejny tytul.}

  You can create overlays\dots
  \begin{itemize}
  \item using the \texttt{pause} command:
    \begin{itemize}
    \item
      First item.
      \pause
    \item    
      Second item.
    \end{itemize}
  \item
    using overlay specifications. \alert{Ponumerowane}:
    \begin{enumerate}
    \item<3->
      First item.
    \item<4->
      Second item.
    \end{enumerate}
  \item
    using the general \texttt{uncover} command:
    \begin{itemize}
      \uncover<5->{\item
        First item.}
      \uncover<6->{\item
        Second item.}
    \end{itemize}
  \end{itemize}
\end{frame}

\begin{frame}{Dlugi tekst}
Tekst prezentacji: Verdana Bold 16/24 pt
Lorem ipsum dolor sit amet, consectetuer adipiscing elit, sed diam nonummy nibh euismod tincidunt ut laoreet dolore magna aliquam erat volutpat. \\

Duis autem vel eum iriure dolor in hendrerit in vulputate velit esse molestie consequat, vel illum dolore eu feugiat nulla facilisis at vero eros et accumsan et iusto odio dignissim qui blandit praesent luptatum zzril delenit augue duis dolore te feugait nulla facilisi. 
\\
Typi non habent claritatem insitam; est usus legentis in iis qui facit eorum claritatem.
\\ 
\end{frame}

\subsection{Cos tam jeszcze}

\begin{frame}{jakis slajd.}

\begin{block}{Tytul bloku}
nsaxkjanskjnxkjsnxakx sdvc
svsv xcz
adscs
\end{block}
\begin{block}{Drugi blok}
askxmksxmsckmaskcnksaxmZ kkcmzx
asxax
acsxz
\end{block}
\end{frame}


\section{Wyniki}

\subsection{Najwazniejsze wyniki}

\begin{frame}{Slajdzik.}
\end{frame}

\subsection{Cos wiecej}

\begin{frame}{Jeszcze jeden Slajd.}
tresc
\end{frame}

\section*{Podsumowanie}

\begin{frame}{Podsumowanie}

  % Keep the summary *very short*.
  \begin{itemize}
  \item
    The \alert{first main message} of your talk in one or two lines.
  \item
    The \alert{second main message} of your talk in one or two lines.
  \item
    Perhaps a \alert{third message}, but not more than that.
  \end{itemize}
  
  % The following outlook is optional.
  \vskip0pt plus.5fill
  \begin{itemize}
  \item
    Outlook
    \begin{itemize}
    \item
      Something you haven't solved.
    \item
      Something else you haven't solved.
    \end{itemize}
  \end{itemize}
\end{frame}

% All of the following is optional and typically not needed. 
\appendix
\section<presentation>*{\appendixname}
\subsection<presentation>*{For Further Reading}

\begin{frame}[allowframebreaks]
  \frametitle<presentation>{For Further Reading}
    
  \begin{thebibliography}{10}
    
  \beamertemplatebookbibitems
  % Start with overview books.

  \bibitem{Author1990}
    A.~Author.
    \newblock {\em Handbook of Everything}.
    \newblock Some Press, 1990.
 
    
  \beamertemplatearticlebibitems
  % Followed by interesting articles. Keep the list short. 

  \bibitem{Someone2000}
    S.~Someone.
    \newblock On this and that.
    \newblock {\em Journal of This and That}, 2(1):50--100,
    2000.
  \end{thebibliography}
\end{frame}

\end{document}