% !TEX encoding = UTF-8 Unicode
\documentclass[a4paper,12pt]{scrbook}

% typesetting
\raggedbottom
\usepackage{xltxtra}
\usepackage{fontspec}
\setmainfont[
  Mapping=tex-text,
  Fractions=On,
  Ligatures={TeX},
  %Contextuals={Swash,Alternate},
  Alternate=1,
  %Style={Historic},
  BoldFont=Calendas_Plus_Bold.otf,
  ItalicFont=Calendas_Plus_Italic.otf
]{Calendas_Plus.otf}
% special fraction characters
\usepackage{xfrac}
% language management
\usepackage[british]{babel}
% quotes management
\usepackage[strict=true,autopunct=true,style=british]{csquotes}
% cite management
\usepackage{cite}
% floats
\usepackage{float}
\usepackage{subcaption}
% lists
\usepackage{enumitem}
% tables
\usepackage{tabularx}
\usepackage[table]{xcolor}
\newcolumntype{L}[1]{>{\hsize=#1\hsize\raggedright\arraybackslash}X}%
\newcolumntype{R}[1]{>{\hsize=#1\hsize\raggedleft\arraybackslash}X}%
\newcolumntype{C}[1]{>{\hsize=#1\hsize\centering\arraybackslash}X}%

\usepackage{lscape}
\usepackage{url}
\usepackage[pdftitle={Identification of the selected types of substructures in the opinion networks},pdfauthor={Michal Ochman},pdfcreator={LaTeX with hyperref},pdfsubject={Identification of the selected types of substructures in the opinion networks.},pdfkeywords={graph theory, graphs, networks, complex networks, data mining, forums, opinions},unicode]{hyperref}

% geometry
\usepackage{geometry}
\geometry{hmargin={2cm, 2cm}, height=10.0in}
% math
\usepackage{amssymb}
\usepackage{amsmath}
% sources
\usepackage{algpseudocode}
\usepackage[chapter]{algorithm}
\usepackage{listings}
\lstset{basicstyle=\scriptsize,breaklines=true,breakatwhitespace=true,captionpos=b,numbers=left,numbersep=10pt,numberstyle=\scriptsize,showspaces=false,showstringspaces=false,stepnumber=1,tabsize=2}

%\renewcommand{\contentsname}{Table of Contents}
%\renewcommand{\bibname}{References}

\pagestyle{plain}

% show counter up to the paragraph level
\setcounter{secnumdepth}{3}
% show in toc up to the paragraph level
\setcounter{tocdepth}{2}

% force \displaystyle for all math in a document
\everymath{\displaystyle}