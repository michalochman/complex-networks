\chapter{Summary}

  One of the biggest challenges was coping with such a huge amount of data. The thesis assumed that it will be possible to find examples of surreptitious messages glorifying or condemning certain products. Unfortunately such examples were never found manually and the whole analysis was performed on a blind assumption that they are there somewhere in hope of finding them. If such messages were available \emph{a priori} then the methodology could have been adjusted appropriately.  Nevertheless, it was very inspiring to conduct such analysis to see what can we find out about the community by exploring the way it communicates.
  
  \ldots
  
  conclusions coming from interpretation of networks acquired using collaborative similarity and motif discovery
  
  \ldots
  
  summary of the complex networks theory as a tool for understanding data as such presented here
  
  \ldots

%  There is, and there will probably never be, an universal way of exploring complex networks.

  \section{What I did}

    \subsection{What worked}

	\subsection{What did not work}

    \section{Potential future work}
    
      \ldots
      
      complete antithesis --- marketing campaign focused on users based on the changes of their product preferences (as well as their nearest neighbour's preferences)
      
      \ldots

    \section{TODO / Other promising approaches I did not have time to tackle}
    
      \ldots
      
      popularity of products based on the keywords appearing in \textquote{long discussions} or in very old, bumped discussions
      
      \ldots
      
      identification of user groups discussing the most and the least popular topics
      
      \ldots
      
      distribution of keywords by the time they take part in discussions
      
      \ldots
      
      response time distributions --- finding the optimal time of operation for marketing purposes
      
      \ldots
      
      investigation of opinion persistence of hubs over time
      
      \ldots