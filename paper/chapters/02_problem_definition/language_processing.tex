\section{Natural language processing}

  \emph{Natural language processing} (\emph{NLP}) is a field of computer science, artificial intelligence, and linguistics concerned with the interactions between computers and human (natural) languages. As such, NLP is related to the area of human–computer interaction. Many challenges in NLP involve natural language understanding -- that is, enabling computers to derive meaning from human or natural language input. The history of NLP generally starts in the 1950s, although work can be found from earlier periods. In 1950, Alan Turing published an article titled \emph{Computing Machinery and Intelligence} which proposed what is now called the Turing test as a criterion of intelligence\cite{Turing1950}.

  \subsection{Text segmentation}

    \emph{Text segmentation} is the process of dividing written text into meaningful units, such as words, sentences, or topics. The term applies both to mental processes used by humans when reading text, and to artificial processes implemented in computers, which are the subject of natural language processing. 
      
    The problem seems trivial in English, where words usually have explicit word boundary markers, such as the word spaces it is certainly non-trivial in other written languages where such signals are sometimes ambiguous and not present. The equivalent to this character is not found in all written scripts, and without it word segmentation is a difficult problem. Languages which do not have a trivial word segmentation process include Chinese, Japanese, where sentences but not words are delimited, Thai and Lao, where phrases and sentences but not words are delimited, and Vietnamese, where syllables but not words are delimited. However---even in written English---it is not as easy as one may think, because users of the forum are neither exclusively native English speakers nor they use completely grammatically correct spelling and punctuation. Despite the fact that in English, the space is a good approximation of a word delimiter, incorrect use of punctuation makes the problem more difficult; there are many edge cases such as hyphenated words, emoticons, and larger constructs such as URIs (which for some purposes may count as single tokens). A classic example is \textquote{New York-based}, which a naive tokenizer may break at the space even though the better break is (arguably) at the hyphen. Some other examples where the space character alone may not be sufficient include contractions like \emph{can't} for \emph{can not}.
      
    The Unicode Consortium has published a Standard Annex\cite{Davis2012} on Text Segmentation, exploring the issues of segmentation in multiscript texts.
      

  \subsection{Word stemming}

    In linguistic morphology and information retrieval, stemming is the process for reducing inflected (or sometimes derived) words to their \emph{stem}, \emph{base} or \emph{root form}---generally a written word form. The stem need not be identical to the morphological root of the word; it is usually sufficient that related words map to the same stem, even if this stem is not in itself a valid root. Algorithms for stemming have been studied in computer science since 1968. Many search engines treat words with the same stem as synonyms as a kind of query broadening, a process called conflation.

    Stemming programs are commonly referred to as stemming algorithms or stemmers.      
    
    \ldots

  \subsection{Word classification}

    \subsubsection{Function words}
    
      \ldots
      
    \subsubsection{Stop words}

      In computing, stop words are words which are filtered out prior to, or after, processing of natural language data (text). There is not one definite list of stop words which all tools use, if even used. Some tools specifically avoid removing them to support phrase search.

      Any group of words can be chosen as the stop words for a given purpose. For some search machines, these are some of the most common, short function words, such as \emph{the}, \emph{is}, \emph{at}, \emph{which}, and \emph{on}. In this case, stop words can cause problems when searching for phrases that include them, particularly in names such as `The Who', `The The', or `Take That'. Other search engines remove some of the most common words—including lexical words, such as "want"—from a query in order to improve performance.

    \subsubsection{Content words (open class)}

      In linguistics content words are words such as nouns, most verbs, adjectives, and adverbs that refer to some object, action, or other non-linguistic meaning. Content words are open class words, meaning that new content words can be added to the lexicon easily, through such processes as compounding, derivation, inflection, coining, and borrowing.
        
      Content words, or lexical words, (including nouns, verbs, adjectives, and most adverbs) are words that carry the content or the meaning of a sentence and are open-class words. Words in open classes (content/lexical words) carry the primary communicative force of an utterance. These words are usually variable in form (inflected), especially in inflecting languages. Their distribution is not definable by the grammar.

      Typical open classes are the class of nouns, the class of verbs, the class of adjectives, and the class of adverbs. However, this varies between languages; for example, in Japanese, pronouns form an open class (the distinction between nouns and pronouns is not always clear in Japanese), while verbs form a closed class.

      Open-class words are not considered part of the core language[citation needed] and as such they can be changed, replaced or dropped from the common lexicon, which can encompass many thousands of them. For living languages, this change is noticeable within an individual lifespan, and usually faster. Closed-class words, on the other hand, are always relatively few and resistant to change. They are unproductively and are generally invariable in form (except demonstratives, modals and some pronouns).
  
  \subsection{Zipf's law}
  
    Zipf's law, an empirical law formulated using mathematical statistics, refers to the fact that many types of data studied in the physical and social sciences can be approximated with a Zipfian distribution, one of a family of related discrete power law probability distributions. The law is named after the American linguist George Kingsley Zipf (1902–1950), who first proposed it (Zipf 1935, 1949), though the French stenographer Jean-Baptiste Estoup (1868-1950) appears to have noticed the regularity before Zipf.\cite{ManningSchutze1999} It was also noted in 1913 by German physicist Felix Auerbach.\cite{Auerbach1913}
    
    \subsubsection{Motivation}
    
      Zipf's law states that given some corpus of natural language utterances, the frequency of any word is inversely proportional to its rank in the frequency table. Thus the most frequent word will occur approximately twice as often as the second most frequent word, three times as often as the third most frequent word, etc. For example, in the Brown Corpus of American English text, the word "the" is the most frequently occurring word, and by itself accounts for nearly 7\% of all word occurrences (69,971 out of slightly over 1 million). True to Zipf's Law, the second-place word \textquote{of} accounts for slightly over 3.5\% of words (36,411 occurrences), followed by \textquote{and} (28,852). Only 135 vocabulary items are needed to account for half the Brown Corpus.
  
      The same relationship occurs in many other rankings unrelated to language, such as the population ranks of cities in various countries, corporation sizes, income rankings, and so on. The appearance of the distribution in rankings of cities by population was first noticed by Felix Auerbach in 1913.\cite{Auerbach1913} Empirically, a data set can be tested to see if Zipf's law applies by running the regression $log R = a - b log n$ where $R$ is the rank of the datum, $n$ is its value and $a$ and $b$ are constants. Zipf's law applies when $b = 1$. When this regression is applied to cities, a better fit has been found with $b = 1.07$. While Zipf's law holds for the upper tail of the distribution, the entire distribution of cities is log-normal and follows Gibrat's law.\cite{Eeckhout2004} Both laws are consistent because a log-normal tail can typically not be distinguished from a Pareto (Zipf) tail.
    
    \subsubsection{Theoretical review}

      Zipf's law is most easily observed by plotting the data on a log-log graph, with the axes being log (rank order) and log (frequency). For example, the word \textquote{the} (as described above) would appear at $x = log(1)$, $y = log(69971)$. The data conform to Zipf's law to the extent that the plot is linear.
      
      Formally, let:
      \begin{itemize}
        \item $N$ be the number of elements;
        \item $k$ be their rank;
        \item $s$ be the value of the exponent characterizing the distribution.
      \end{itemize}
      Zipf's law then predicts that out of a population of $N$ elements, the frequency of elements of rank $k\mbox{,} f(k;s,N)$, is:
      \begin{equation}
        f(k;s,N)=\frac{\sfrac{1}{k^s}}{\sum\limits_{n=1}^N \frac{1}{n^s}}\mbox{.}
      \end{equation}

      Zipf's law holds if the number of occurrences of each element are independent and identically distributed random variables with power law distribution $p(f) = \alpha f^{-1-\sfrac{1}{s}}$.\cite{AdamicHuberman2002}

      In the example of the frequency of words in the English language, $N$ is the number of words in the English language and, if we use the classic version of Zipf's law, the exponent $s$ is $1$. $f(k; s,N)$ will then be the fraction of the time the $k$th most common word occurs.

      The law may also be written:
      \begin{equation}
        f(k;s,N)=\frac{1}{k^s H_{N,s}}\mbox{,}
      \end{equation}
      where $H_{N,s}$ is the $N$th generalized harmonic number.
The simplest case of Zipf's law is a \textquote{$\sfrac{1}{f}$ function}. Given a set of Zipfian distributed frequencies, sorted from most common to least common, the second most common frequency will occur $\sfrac{1}{2}$ as often as the first. The third most common frequency will occur $\sfrac{1}{3}$ as often as the first. The $n$th most common frequency will occur $\sfrac{1}{n}$ as often as the first. However, this cannot hold exactly, because items must occur an integer number of times; there cannot be $2.5$ occurrences of a word. Nevertheless, over fairly wide ranges, and to a fairly good approximation, many natural phenomena obey Zipf's law.

      Mathematically, the sum of all relative frequencies in a Zipf distribution is equal to the harmonic series, and
      \begin{equation}
        \sum_{n=1}^\infty \frac{1}{n}=\infty\mbox{.}
      \end{equation}
      In human languages, word frequencies have a very heavy-tailed distribution, and can therefore be modeled reasonably well by a Zipf distribution with an $s$ close to $1$.

      As long as the exponent $s$ exceeds $1$, it is possible for such a law to hold with infinitely many words, since if $s > 1$ then
      \begin{equation}
        \zeta (s) = \sum_{n=1}^\infty \frac{1}{n^s}<\infty\mbox{,}
      \end{equation}
      where $\zeta$ is Riemann's zeta function.

    \subsubsection{Statistical explanation}
      
      It is not known why Zipf's law holds for most languages.\cite{Brillouin2004} However, it may be partially explained by the statistical analysis of randomly generated texts. Wentian Li has shown that in a document in which each character has been chosen randomly from a uniform distribution of all letters (plus a space character), the \textquote{words} follow the general trend of Zipf's law (appearing approximately linear on log-log plot).\cite{WentianLi1992} Vitold Belevitch in a paper, \emph{On the Statistical Laws of Linguistic Distribution} offered a mathematical derivation. He took a large class of well-behaved statistical distributions (not only the normal distribution) and expressed them in terms of rank. He then expanded each expression into a Taylor series. In every case Belevitch obtained the remarkable result that a first-order truncation of the series resulted in Zipf's law. Further, a second-order truncation of the Taylor series resulted in Mandelbrot's law.\cite{Neumann2013,Belevitch1959}

      Zipf himself proposed that neither speakers nor hearers using a given language want to work any harder than necessary to reach understanding, and the process that results in approximately equal distribution of effort leads to the observed Zipf distribution.\cite{Zipf1949,CanchoSole2003}
  
  \subsection{Heaps' law}
  
    In linguistics, Heaps' law (also called Herdan's law) is an empirical law which describes the number of distinct words in a document (or set of documents) as a function of the document length (so called type-token relation). It can be formulated as
    \begin{equation}
      V_R(n) = Kn^\beta\mbox{,}
    \end{equation}
    where $V_R$ is the number of distinct words in an instance text of size $n$. $K$ and $\beta$ are free parameters determined empirically. With English text corpora\footnote{A corpus (pl. corpora) or text corpus is a large and structured set of texts. They are used to do statistical analysis and hypothesis testing, checking occurrences or validating linguistic rules within a specific language territory.}, typically $K$ is between $10$ and $100$, and $\beta$ is between $0.4$ and $0.6$.

    The law is frequently attributed to Harold Stanley Heaps, but was originally discovered by Gustav Herdan in 1960.\cite{Egghe2007} Under mild assumptions, the Herdan–Heaps law is asymptotically equivalent to Zipf's law concerning the frequencies of individual words within a text.\cite{Kornai1999} This is a consequence of the fact that the type-token relation (in general) of a homogenous text can be derived from the distribution of its types.\cite{Milicka2009}

    Heaps' law means that as more instance text is gathered, there will be diminishing returns in terms of discovery of the full vocabulary from which the distinct terms are drawn.

    It is interesting to note that Heaps' law also applies to situations in which the \textquote{vocabulary} is just some set of distinct types which are attributes of some collection of objects. For example, the objects could be people, and the types could be country of origin of the person. If persons are selected randomly (that is, we are not selecting based on country of origin), then Heaps' law says we will quickly have representatives from most countries (in proportion to their population) but it will become increasingly difficult to cover the entire set of countries by continuing this method of sampling.