\section{Network theory}

	\subsection{Types of networks}
		
	\subsection{Real world networks}
		
	\subsection{Networks properties}
		
		\subsubsection{Small-world effect}
			
		\subsubsection{Transitivity}
			In networks it is quite common to find that if vertex A is is connected to vertex B and vertex B is connected to vertex C, then the probability that vertex A is connected to vertex C is higher than the probability that vertex A is connected to any other randomly chosen vertex. For example, in social networks, the friends of your friends are more likely to also be your friends than a random person that is not a friend of your friend. This deviation from the behaviour of random graphs can be seen in a network property called transitivity (also known as clustering, although this term has another meaning which may be confusing). From topological point of view, transitivity means that there are more sets of three vertices, each of which is connected to the other two -- called triangles -- in the network is heightened. It may be quantified by defining a clustering coefficient $C$ thus:
			\[ C = \frac{3 \times \mbox{number of triangles in the network}}{\mbox{number of connected triples of vertices}} \mbox{,} \]
			where a "connected triple" means a single vertex with edges running to an unordered pair of others.
			
		\subsubsection{Degree distribution}
			
		\subsubsection{Resilience}
			
		\subsubsection{Mixing patterns}
			
		\subsubsection{Degree correlations}
		
		\subsubsection{Communities}
			
		\subsubsection{Other properties}
		
	\subsection{Networks growth}
		
	\subsection{Processes taking place on networks}