\chapter{Introduction}

	\section{Introduction}

		In this work I will consider the problem of identifying different types of structures and substructures in opinion networks. The formal definition for the term "opinion network" is provided in Chapter \ref{chap:definitions}, but informally we can describe it as a complex network representing opinions amongst a set of particular agents, e.g. individual users of message boards.
		
	
	\section{Motivation}

		Due to the hypothesis that some agents may be dishonest and may, for example, create multiple accounts on message boards to either advertise and/or malign certain brands of products it is desirable, if not necessary, to filter out quasi-opinions posted by such agents. For many message boards, like the one examined during writing of this thesis, honest and unpolluted opinions are key factor in retaining its reputability. Without the possibility to identify and remove unwanted opinions it's easy to decline and lose the user base. However, one has to note that there is no sure and reliable way of doing it automatically or applying it in a generic way across different opinion networks.
		
	\section{Related Work}
		